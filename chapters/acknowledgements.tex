This thesis and the Ph.D journey was not available without the help of so many people, which I would like to acknowledge only a few.

First of all, I would like to show my largest gratitude to my advisor Prof. Hiroki Ishikuro.
Entering the lab group, I hardly new anything about circuit design and researches but he led me up patiently and step-by-step. Interestingly, the first thing he taught us was Opamp design, and distilling that initial knowledge, Opamp design became the core part of this thesis. I would like to thank that he gave me various research opportunities, for a number of tapeouts and more on interacting with other research groups. The experience upon conducting researches with the Extremely Low-Power (ELP) group was very valuable, given feedbacks from industrial specialists. Also opportunities with collaborating with Fujitsu was very fortunate as well.
% 回路設計や研究への道を初歩から教えてもらった。

Not only Prof. Ishikuro taught me how to design circuits and publish papers at international conferences, but to enjoy and get most out of academic events as well (e.g. looking for the best local foods..). I really remember the first time when we visited the bay area for a conference (CICC), and Prof. Ishikuro set up opportunities to discuss with researchers at Apple, Stanford, UCB and imec. It was my first time interacting with top researchers overseas, and gave me high motivations to compete and collaborate with them in the near future. Recalling, those excitements led me to research experiences at Stanford over the following years as well.

I am very thankful to Prof. Tadahiro Kuroda, who initially taught me the spirit to challenge to top researchers and universities.
When I was still a undergrad and deciding which department to proceed to, I heard Prof. Kuroda's talk which was about the journey upon competing with the world's top universities, and the importance of publishing researches at the premiere conferences (ISSCC). Such vision inspired me and drove my life towards Hardware design and researches. %After 10 years from hearing that talk, I was able to publish a few papers to such conferences as well.

% committee members
I would like to thank my committee members, Prof. Nobuhiko Nakano and Prof. Tetsuya Iizuka who had generously taken their time for this Ph.D dissertation and would like to thank their guidance and advises to this thesis. 

% lab
I would like to thank my lab members of Ishikuro/Kuroda group (just to name a few: Yuki Urano, Yuya Hasega, Teruo Jo, Atsutake Kosuge, Haruki Fukuda, Teturo Ogaki, Katuki Ohata) whom worked through countless sleep-less nights during number of tapeouts. I don't think any chip would have worked without their help and encouragements. Especially Dr. Akira Shikata helped me establish the knowledge of low-power SAR ADCs with his deep insights. Ryota Sekimoto and Takashi Chiba taught me patiently about the basics of data-converter designs. 

% STARC
I would like to show gratitude to members of the Extremely Low-Power (ELP) group, (especially Yasuyuki Hiraku and Isamu Hayashi) who passionately discussed  and taught me basic flows and rules of IC designs. Also, I would like to thank members of Fujitsu Lab. (Sanroku Tsukamoto and Masato Yoshioka) who have given so many deep-insights of state-of-the-art ADC designs and various feedback to my research. 

% Toshiba
I would like to thank number of colleagues at Toshiba who have always given me generous supports. 
Firstly, Hirotomo Ishii, Tomohiko Sugimoto, Daisuke Kurose, and Naoya Waki have alwys been a respected analog designer, who taught me patiently about ADC design from the very basics. From them, I was able to learn how product-level design differs from research-level designs and what it is to become a professional analog designer.

Moreover, I would like to thank Masanori Furuta and Akihide Sai for mentoring through the research projects I have gone through at Toshiba R&D. Industry driven researches differs greatly from academic researches, and I was able to learn a lot from the way they handled emerging research projects. Most of the researches within Toshiba would not have been accomplished at all if they were not my boss.

% stanford
I would like to thank Prof. Mark Horowitz and his students for their generosity and kindness during my stay at Stanford. Mark patiently mentored me through the research I was going through, I would like to thank for his supportiveness. Learning the basics of computer architectures and hardware-software co-design was a great honor and simultaneously a great experience. Also I would like to thank Edward Lee, who had been an awesome collaborator at Stanford! 

% family
Last but not least, I would like to show great gratitude to my Mom and Dad, who have always been the most closest  supporters of my researches and careers. I was very lucky given the amount of opportunities and generous educations they have given to me (including the 4.5 year life at the US), which is a key piece showing what I am now. I would also like to thank my partner Sayaka, who had been supportive both in both professional and private lives.
