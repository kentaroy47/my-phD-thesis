Along with CMOS scaling, wireless/wireline communication performances have greatly advanced.
To realize a system on chip (SoC) for such products, high-performance analog circuits are necessary; for example, high-speed and high-precision analog-to-digital converters (ADCs) are often required to convert the received analog signal to digital. 
While such SoCs utilize the most leading CMOS technologies to cut down the costs of the digital circuits, the analog circuit performance inconveniently degrades as the CMOS scaling advance. To name an example, the Opamp gain performance greatly degrades with scaling with worsened transistor gain and lower supply voltages.
On the contrary, as the communication standards further evolve, the performance demands toward analog circuits continue to increase.
Thus, the design of ADCs in scaled CMOS process environments become one of the most challenging and critical fields of circuit design.

In this thesis, we aim to explore Hybrid ADCs utilizing successive-approximation (SA) circuitry, which can benefit from process scaling. And ultimately, we target to establish an ADC design methodology suitable for scaled CMOS technologies. In chapter 1, the technology trends of the CMOS process scaling are discussed and scaling effects to the analog circuitry are studied. Moreover, we show that SA circuitry is suitable for scaled CMOS and explore its limitations as well. Finally, recent research trends of Hybrid ADCs and its design challenges are discussed.

We propose a Hybrid ADC which heavily utilizes the SA circuitry in chapters 2 and 3. In chapter 2, the Digital Amplifier (DA) technique is proposed to realize power-efficient and accurate amplification in scaled CMOS which utilizes an SA circuitry for amplification. DA cancels out all errors of the low-gain amplifier by feedback based on SA. Moreover, the amplification accuracy can be arbitrary set by configuring the number of bits of the DA; the amplifier gain is decoupled from the transistor intrinsic gain and brings in a new design paradigm for amplifier design in scaled CMOS. The fabricated ADC with DA achieves SNDR of 61.1dB, FoM of 12.8fJ/conv., which is over 3$\times$ improvement compared to conventional ADCs.

In chapter 3, we explore power-efficient and process scalable ultra-high-speed ADCs, required for high-capacity wireless communications.
To achieve low-power and high-speed ADCs, we propose to dynamically configure the ADC architecture reflecting the ADC clock frequency, which we name Dynamic Architecture and Frequency Scaling (DAFS).
The ADC architecture is reconfigured between successive-approximation and flash every clock cycle, relying on the conversion delay. 
A prototype subranging ADC is fabricated in 65 nm CMOS, which is 2$\times$ more power-efficient than the previously reported subranging ADCs.

In chapter 4, we propose a comparator with a variable threshold to explore multi-bit/step comparisons, which can significantly speed up the successive-approximation circuitry implemented in chapters 2 and 3. Finally, we establish a conclusion in chapter 5.